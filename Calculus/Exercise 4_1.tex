\documentclass{article}
\usepackage{amsmath}
\usepackage{amsthm}
\usepackage{amssymb}
\usepackage{enumerate}
\usepackage{graphicx}
% done 
\begin{document}
    \title{MA4\_1 Exercise}
    \author{Wang Yue from CS Elite Class}
    \date{\today}

    \maketitle

    \section*{Exercise 4.2}

    \subsection*{15. Let$f(x) = (x - 3)^{-2}$. Show that there is no value of $c$ in $(1, 4)$ such that $f(4) - f(1) = f'(c)(4 - 1)$. Why does this not contradict the Mean Value Theorem?}

    $\because f'(x) = \frac{-2}{(x - 3)^3}, x \in (-\infty, 3) \cup (3, \infty)$

    $\therefore$ 
    
    when $1 < x < 3$, $f'(x) > \frac 1 4$

    when $3 < x < 4$, $f'(x) < -2$

    $\because \frac{f(4) - f(1)}{4 - 1} = \frac{1 - \frac 1 4}{3} = \frac 1 4$

    $\therefore $ when $1 < x < 4$, there is no value of $c$ in $(1, 4)$ such that $f(4) - f(1) = f'(c)(4 - 1)$.

    This does not contradict the Mean Value Theorem because $f(x)$ is not continuous at $[1, 4]$.

    \subsection*{18. $x^3 + e^x = 0$}

    Let $f(x) = x^3 + e^x$, and obviously $$f'(x) = 3x^2 + e^x$$

    $\because f(-1) = -1 + \frac 1 e < 0, f(1) = 1 + e > 0, f(-1)f(1) < 0$

    $\therefore$ there exists at least one real root of $f(x) = 0$ as $x \in (-1, 1)$

    Suppose there exists two real roots $x_1, x_2$ of $f(x) = 0$.

    $\because f(x)$ is continuous and differentiable on $(x_1, x_2)$, $f(x_1) = f(x_2)$

    $\therefore \exists \xi \in (x_1, x_2)$, such that $$f'(\xi) = e^{\xi} + 3\xi ^2 = 0$$

    $\because e^x > 0, x^2 \geq 0$

    $\therefore f'(\xi) > 0$, which contradicts $\exists \xi \in (x_1, x_2), f'(\xi) = 0$.

    $\therefore$ there is exactly one real root of the equation $x^3 + e^x = 0$

    \subsection*{27. Show that $\sqrt{1 + x} < 1 + \frac 1 2 x$ if $x > 0$.}

    \begin{proof}
        Let $f(x) = \sqrt{1 + x}$, and obviously $$f'(x) = \frac{1}{2\sqrt{1 + x}} < \frac 1 2$$

        By the Lagrange Mean Value Theorem on $(0, 1)$, $\exists \xi \in (0, x)$ such that $$\sqrt{1 + x} - \sqrt{1 + 0} = f'(\xi)(x - 0)$$

        $\therefore \exists \xi \in (0, x)$, such that $$\sqrt{1 + x} - 1 = f'(\xi)x < \frac 1 2 x$$

        $\therefore \sqrt{1 + x} < 1 + \frac 1 2 x$ if $x > 0$
    \end{proof}

    \subsection*{30. If $f'(x) = c$ for all $x$, use Corollary 7 to show that $f(x) = cx + d$ for some constant $d$.}

    \begin{proof}
        Let $g(x) = cx$, so obviously $$g'(x) = c = f'(x)$$

        So by the Corollary 7, $\forall x \in R$, $f(x) = g(x) + d$ where $d$ is a constant.

        It is equivalent to $f(x) = cx + d$ for some constant $d$.
    \end{proof}

    \subsection*{36. A number $a$ is called a \textbf{fixed point} of a function $f$ if $f(a) = a$. Prove that if $f'(x) \not = 1$ for all real numbers $x$, then $f$ has at most one fixed point.}

    \begin{proof}
        We will prove it by contradiction.

        Suppose that $\exists a, b \in R$ such that $$f(a) = a, f(b) = b$$

        By the Lagrange Mean Value Theorem, $\exists \xi \in (a, b)$ such that $$f(b) - f(a) = b - a = f'(\xi) (b - a) $$

        $\therefore \exists \xi \in (a,b)$ such that $f(\xi) = 1$, which contradict the hypothesis that $\forall x, f'(x) \not = 1$

        $\therefore f$ has at most one fixed point.
    \end{proof}

\end{document}