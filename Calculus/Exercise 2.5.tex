\documentclass{article}
\usepackage{amsmath}
\usepackage{amsthm}
\usepackage{amssymb}
\usepackage{enumerate}
\usepackage{graphicx}

\begin{document}
    \title{Exercise 2.5}
    \author{Wang Yue from CS Elite Class}
    \date{\today}

    \maketitle

    \subsection*{37. $\lim_{x \to 1}e^{x^2-x}$}

    Let $f(x) = e^x, g(x) = x^2 - x$, and $f(x)$ and $g(x)$ are continuous.

    By the property of continuous composite function,
    $$
    \begin{aligned}
        \lim_{x \to 1}e^{x^2 - x} &= \lim_{x \to 1}f(g(x)) \\
        &= f(\lim_{x \to 1}g(x)) \\
        &= f(g(1)) = f(0) = 1
    \end{aligned}
    $$

    \subsection*{38. $\lim_{x \to 2} \arctan(\frac{x^2 - 4}{3x^2 - 6x})$}

    Obviously, $y = \arctan x$ and $y = \frac{x^2 - 4}{3x^2 - 6x}$ are continuous.

    So by the property of continuous composite function,

    $$
    \begin{aligned}
        \lim_{x \to 2} \arctan(\frac{x^2 - 4}{3x^2 - 6x}) &= \arctan(\lim_{x \to 2}\frac{(x + 2)(x - 2)}{3x(x - 2)}) \\
        &= \arctan(\lim_{x \to 2}\frac{x + 2}{3x}) \\
        &= \arctan(\frac{2}{3})
    \end{aligned}
    $$

    \subsection*{46. Find the values of $a$ and $b$ that make $f$ continuous everywhere. }

    To make $f(x)$ continuous everywhere, we must satisfy:

    \begin{equation*}
        \left\{
        \begin{aligned}
            \lim_{x \to 2^-}f(x) = f(2) \\
            \lim_{x \to 3^-}f(x) = f(3)
        \end{aligned}
        \right .
    \end{equation*}
    $\because \lim_{x \to 2^-}f(x) = \lim{x \to 2^-}\frac{(x + 2)(x - 2)}{x - 2} = \lim_{x \to 2^-}(x + 2) = 4$

    $\because \lim_{x \to 3^-}f(x) = \lim_{x \to 3^-}(ax^2 - bx + 3) = 9a - 3b + 3$

    $\therefore$ we get the equations:

    \begin{equation*}
        \left \{
        \begin{aligned}
            4a - 2b + 3 = 4 \\
            6 - a + b = 9a - 3b + 3
        \end{aligned}
        \right .
    \end{equation*}

    The solution is $\left \{ \begin{aligned}
        a = \frac 1 2 \\ b = \frac 1 2
    \end{aligned} \right .$


    \subsection*{53. $e^x = 3 - 2x, \quad (0, 1)$}

    Let $f(x) = e^x + 2x - 3$

    $\because f(0) = 1 + 0 - 3 = -2 < 0, f(1) = e + 2 - 3 = e - 1 > 0$

    $\because f(x)$ is continuous, $f(0)f(1) < 0$

    $\therefore$ by the Intermediate Value Theorem, $\exists \xi \in (0, 1), s.t.$ $$f(\xi) = 0  \iff e^\xi = 3 - 2\xi$$

    \subsection*{54. $\sin x = x^2 - x, \quad (1, 2)$}

    Let $f(x) = x^2 - x - \sin x$

    $\because f(1) = 1 - 1 - \sin 1 < 0, f(2) = 4 - 2 - \sin 2 > 2 - 1 > 0$

    $\because f(x)$ is continuous and $f(1)f(2) < 0$

    $\therefore$ by the Intermediate Value Theorem, $\exists \xi \in (1, 2), s.t.$ $$f(\xi) = 0 \iff \sin \xi = \xi ^2 - \xi$$

    \subsection*{61. Prove that cosine is a continuous function.}

    \begin{proof}
        We will prove it by the contradiction.

        Suppose that cosine is a discontinuous function, so $\exists x_0 \in R, s.t.$ $$\lim_{x \to x_0}\cos x \not = \cos x_0$$

        But $\forall \epsilon > 0, \exists \delta = \epsilon$.

        When $\delta > 0$, if $0 < |x - x_0| < \delta$, then 
        
        $$
        \begin{aligned}
            |\cos x - \cos x_0| &= |\cos (\frac{x + x_0}{2} + \frac{x - x_0}{2}) - \cos(\frac{x + x_0}{2} - \frac{x - x_0}{2}) | \\
            &= |-2\sin \frac{x + x_0}{2} \frac{x - x_0}{2} | \\
            &= 2|\sin \frac{x + x_0}{2} \frac{x - x_0}{2} | \\
            &\leq 2 |\sin \frac{x - x_0}{2} | \\
            &< 2 |\frac{x - x_0}{2}| = |x - x_0| = \delta = \epsilon
        \end{aligned}
        $$

        The calculation above indicates that $\lim_x{x \to x_0}\cos x = \cos x_0$, which contradicts with the hypothesis.

        So we can prove that the cosine function always satisfies $\lim_x{x \to x_0}\cos x = \cos x_0$, and it's continuous.

    \end{proof}

    \subsection*{67. Show that the function $$f(x) = \left\{ \begin{array}{ll} x^2\sin(1/x) & \textrm{if $x \not = 0$} \\ 0 & \textrm{if $x = 0$} \end{array} \right.$$ is continuous on $(-\infty, \infty)$}

    $\because$ Obviously, $y = \frac 1 x, y = \sin x$ are continuous functions,

    $\therefore y = \sin(1/x)$ is continuous function.

    $\because y = x^4$ is also continuous function,

    $\therefore f(x) = x^4\sin(1/x), \quad x \not = 0$ is continuous function on $\{x \in R|x \not = 0\}$

    $\because x^4$ is an infinitesimal as $x \to 0$, and $\sin(1/x)$ is bounded on $[-1,1]$ as $x \to 0$,

    $\therefore$ $$\lim_{x \to 0}x^4\sin(1/x) = 0 = f(0)$$

    $\therefore f(x)$ is continuous on $(-\infty, \infty)$

\subsection*{68.(a)}

\begin{proof}
    $\because$ Obviously, $$F(x) = \left \{ \begin{array}{ll}
        x & \textrm{if $x > 0$} \\
        -x & \textrm{if $x < 0$} \\
        0 & \textrm{if $x = 0$}
    \end{array} \right.$$

    $therefore$ When $x \not = 0$, $f(x)$ is continous on $(-\infty, 0)$ and $(0, \infty)$

    $\because \lim_{x \to 0^-}F(x) = \lim_{x \to 0^+}F(x) = 0 = F(0)$

    $\therefore F(x)$ is continuous everywhere.
\end{proof}

\subsection*{86.(b)}

\begin{proof}
    If $f(x) \geq 0$, $|f(x)| = f(x)$ is also continuous.

    If $f(x) \leq 0$, $|f(x)| = -f(x)$ is also continuous.

    If $\exists x_0, s.t. f(x_0) = 0$, then we discuss about a common situation: $|f(x)|$ on $(a,b), x_0 \in (a,b)$

    Without loss of generality, let $f(x) > 0$ as $x \in (a,x_0)$, $f(x) < 0$ as $x \in (x_0, b)$, $f(x) = 0$ as $x = x_0$

    If we can prove in such a common situation $|f(x)|$ is continuous, then we can conclude that $|f(x)|$ is continuous in the domain of $f(x)$.

    Obviously, when $x \in (a, x_0), |f(x)| = f(x)$ is continous, so is $x \in (x_0, b)$.

    $\because \lim_{x \to x_0^-}|f(x)| = \lim_{x \to x_0^+}|f(x)| = 0 = f(x_0)$
    
    $\therefore |f(x)|$ is continuous on $(a,b)$

    $\therefore$ $|f(x)|$ is continuous.
\end{proof}

\subsection*{68.(c)}

No. A counterexample is $f(x) = \left \{ \begin{array}{ll}
    x - 2 & \textrm{if $x < 0$} \\
    x + 2 & \textrm{if $x \geq 0$}
\end{array} \right.$

In this case, when $x < 0$, $|f(x)| = |x - 2| = 2 - x$, $|f(x)| = |x| + 2$, which is obviously continuous.

However, the original function $f(x)$ is not continuous at $x = 0$, for $$\lim_{x \to 0^-}f(x) = -2, \lim_{x \to 0^+}f(x) = 2$$
\end{document}