\documentclass{article}
\usepackage{amsmath}
\usepackage{amsthm}
\usepackage{amssymb}
\usepackage{enumerate}
\usepackage{graphicx}

\begin{document}
    \title{Exercise 2.6 Homework}
    \author{Wang Yue from CS Elite Class}
    \date{\today}

    \maketitle

    \section*{Evaluate the limit and justify each step by indicating the appropriate properties of limits.}

    \subsection*{14. $\lim_{x \to \infty}\sqrt{\frac{12x^3 - 5x + 2}{1 + 4x^2 + 3x^3}}$}

    $$
    \begin{aligned}
        \lim_{x \to \infty}\sqrt{\frac{12x^3 - 5x + 2}{1 + 4x^2 + 3x^3}} &= \lim_{x \to \infty}\sqrt{\frac{12 - \frac{5}{x^2} + \frac{2}{x^3}}{3 + \frac{4}{x^2} + \frac{1}{x^3}}} \\
        &= \frac{12 - 0 + 0}{3 + 0 + 0} \\
        &= 4
    \end{aligned}
    $$

    \section*{Find the limit or show that it does not exist.}

    \subsection*{26. $\lim_{x \to -\infty}(x + \sqrt{x^2 + 2x})$}

    $$
    \begin{aligned}
        \lim_{x \to -\infty}(x + \sqrt{x^2 + 2x}) &= \lim_{x \to -\infty}\frac{x^2 - x^2 - 2x}{x - \sqrt{x^2 + 2x}} \\
        &= \lim_{x \to -\infty}\frac{2x}{\sqrt{x^2 + 2x} - x} \\
        &= \lim_{x \to -\infty}\frac{-2}{\sqrt{1 - \frac 2 x} + 1} \\
        &= \frac{-2}{1 + 1} = -1
    \end{aligned}
    $$

    \subsection*{27. $\lim_{x \to \infty}(\sqrt{x^2 + ax} - \sqrt{x^2 + bx})$}

    $$
    \begin{aligned}
        \lim_{x \to \infty}(\sqrt{x^2 + ax} - \sqrt{x^2 + bx}) &= \lim_{x \to \infty}\frac{x^2 + ax - x^2 - bx}{\sqrt{x^2 + ax} + \sqrt{x^2 + bx}} \\
        &= \lim_{x \to \infty}\frac{(a-b)x}{\sqrt{x^2 + ax} + \sqrt{x^2 + bx}} \\
        &= \lim_{x \to \infty}\frac{a - b}{\sqrt{1 + \frac 1 a} + \sqrt{1 + \frac 1 b}} \\
        &= \frac{a - b}{1 + 1} \\
        &= \frac{a - b}{2}
    \end{aligned}
    $$

    \subsection*{28. $\lim_{x \to \infty}\sqrt{x^2 + 1}$}

    The limit does not exist. Here is the reason:

    $\forall \epsilon > 0, \exists N = \lceil \sqrt{\epsilon ^2 - 1} \rceil$

    if $ x > N$, then $$\sqrt{x^2 + 1} > \sqrt{N^2 + 1} > \epsilon$$

    So $$\lim_{x \to \infty}\sqrt{x ^2 + 1} = \infty$$ which means that the limit does not exist.

    \subsection*{30. $\lim_{x \to \infty}(e^{-x} + 2 \cos 3x)$}

    The limit does not exist, and here is the reason:

    Obviously, $\lim_{x \to \infty}e^{-x} = 0$.

    Let $a_n = \frac{2n\pi}{3}, b_n = {(2n + 3)\pi}{3}$,

    and we know $a_n \to \infty, b_n \to \infty$ as $n \to \infty$

    By the Henie Theorem, $$\lim_{x \to \infty}2\cos 3x = \lim_{x \to \infty}2 \cos 2n \pi = 0$$

    $$lim_{x \to \infty}2\cos 3x = \lim_{x \to \infty}(2n\pi + \pi) = -1$$

    For these two limits are not equal, $\lim_{x \to \infty}2\cos 3x$ does not exist.

    So $\lim_{x \to \infty}{e^{-x} + 2 \cos 3x}$ does not exist.

    \subsection*{33. $\lim_{x \to -\infty}\arctan(e^x)$}

    Let $u = e^x$, $u_0 = \lim_{x \to -\infty}e^x = 0$. So $$\lim_{x \to -\infty}\arctan(e^x) = \lim_{u \to u_0}\arctan(u) = 0$$

    \subsection*{36. $\lim_{x \to \infty}\frac{\sin^2 x}{x^2 + 1}$}

    Obviously, $$\lim_{x \to \infty} \frac{1}{x^2 + 1} = 0$$ which means $\lim_{x \to \infty}\frac{1}{x^2 + 1}$ is infinitesimal.

    And $\because$ $$\lim_{x \to \infty}\sin ^2 x \in [0, 1]$$ which means $\lim_{x \to \infty}\sin^2 x$ is bounded.

    By the properties of infinitesimal, $$\lim_{x \to \infty}\frac{\sin^2 x}{x ^2 + 1} = 0$$

    \subsection*{37. $\lim_{x \to \infty}(e^{-2x} \cos x)$}

    Obviously, $$\lim_{x \to \infty} e^{-2x} = 0$$ which means this value is infinitesimal.

    And $\because$ $$\lim_{x \to \infty}\cos x \in [-1, 1]$$ which means this value is bounded.

    By the properties of infinitesimal, $$\lim_{x \to \infty}e^{-2x}\cos x = 0$$

    \subsection*{51. A function $f$ is a ratio of quadratic functions and has a vertical asymptote $x = 4$ and just one x-intercept, $x=1$. It is known that $f$ has a removable discountinuity at $x = -1$ and $\lim_{x \to -1}f(x) = 2$. Evaluate}

    \begin{enumerate}[(a)]
        \item $f(0)$
        
        Let $f(x) = \frac{g(x)}{h(x)}$.

        $\because f(x)$ has a vertical asymptote $x = 4$

        $\therefore h(x) = 0$ as $ x = 4$, which means $$h(x) = (x - 4)^2$$

        And we can establish a system of equations:

        $$
        \begin{cases}
            f(1) = \frac{g(1)}{h(1)} = 0 \\
            f(-1) = \frac{g(-1)}{h(-1)} = 2
        \end{cases}
        $$

        Solving this system, we can get $g(x) = x(x - 1)$

        Therefore, $f(0) = 0$

        \item $\lim_{x \to \infty}f(x)$
        $$
        \begin{aligned}
            \lim_{x \to \infty}f(x) &= \lim_{x \to \infty}\frac{x(x - 1)}{(x - 4)^2} \\
            &= \lim_{x \to \infty}\frac{1 - \frac 1 x}{(1 - \frac 4 x) ^2} \\
            &= \frac{1 - 0}{(1 - 0)^2} \\
            &= 1
        \end{aligned}
        $$
    \end{enumerate}

    \subsection*{71. Use Definition 8 to prove that $\lim_{x \to -\infty}\frac 1 x = 0$.}

    \begin{proof}
        $\forall \epsilon > 0, \exists N = -\frac{1}{\epsilon}$

        If $x < N$, then $$|f(x) - 0| = -\frac 1 x  < -\frac 1 N = \epsilon$$


    \end{proof}


    \subsection*{75. Prove that $$\lim_{x \to \infty}f(x) = \lim_{t \to 0^+}f(1/t)$$ and $$\lim_{x \to -\infty}f(x) = \lim_{t  \to 0^-}f(1/t)$$ if these limits exist.}
     
    \begin{proof}
        $\because \lim_{x \to \infty}x = \infty =  \lim_{t \to 0^+}\frac 1 t = \infty$
        
        $\therefore \lim_{x \to \infty}f(x) = \lim_{t \to 0^+}f(1/t)$

        $\because \lim_{x \to -\infty}x = -\infty = \lim_{x \to 0^-}\frac 1 t = -\infty$
        
        $\therefore \lim_{x \to -\infty}f(x) = \lim_{t\to 0^-}f(1/t)$
        
    \end{proof}

\end{document}