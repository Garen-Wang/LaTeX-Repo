\documentclass{article}
\usepackage{amsmath}
\usepackage{amsthm}
\usepackage{amssymb}
\begin{document}
    \title{Exercise 11.1}
    \author{Wang Yue from CS Elite Class}
    \date{\today}
    \maketitle

    \subsection*{23\textasciitilde 47. Determine whether the sequence converges or diverges. If it converges, find the limit.}

    \subsection*{23. $a_n = 1 - (0.2) ^ n$}
        The sequence converges, and the limit is 1.
    \begin{proof}

        proof(1) :$$a_{n + 1} - a_n = 1 - (0.2)^{n - 1} - 1 + (0.2)^n = (1 - 0.2) \times (0.2)^n > 0$$

        so $a_n$ is increasing.

        $$a_n = 1 - (0.2)^n < 1$$

        so $a_n$ is bounded above.

        By the Monotonic Sequence Theorem, the sequence is convergent.

        $$\lim_{n \to \infty}a_n = \lim_{n \to \infty}1 - \lim_{n \to \infty}(0.2)^n = 1 - 0 = 1$$

        so the limit of the sequence is $1$.

    \end{proof}
    \subsection*{29. $a_n = \tan(\frac{2n\pi}{1 + 8n})$}
    The sequence converges,  and the limit is also $1$.
    \begin{proof}
        let $b_n = \frac{2n\pi}{1  + 8n}$, so $$\lim_{n \to \infty}b_n = \lim_{n \to \infty}\frac{2\pi}{8 + \frac{1}{n}} = \frac{\pi}{4}$$
        

        Let $L = \frac{\pi}{4}$, and by the Henie Theorem, we have $$\lim_{n \to \infty} a_n = \lim_{x \to L}\tan x = 1$$
    \end{proof}
    
    \subsection*{30. $a_n = \sqrt{\frac{n+1}{9n+1}}$}
    The sequence converges, and its limit is $1$.
    \begin{proof}
        $$
        \begin{aligned}
            \lim_{n \to \infty}{a_n} &= \sqrt{\lim_{n \to \infty}\frac{n+\frac{1}{9} + \frac{8}{9}}{9n+1}} \\
            &= \sqrt{\lim_{n \to \infty} (\frac{1}{9} + \frac{8}{9(9n+1)})} \\
            &= \sqrt{ (\lim_{n \to \infty}\frac{1}{9} + \lim_{n \to \infty}\frac{8}{9(9n+1)})} \\
            &= \sqrt{\frac{1}{9}}  \\
            &= \frac{1}{3}
        \end{aligned}
        $$
    \end{proof}

    \subsection*{32. $a_n = e^{2n / (n + 2)}$}
    The sequence converges, and the limit is $e^2$.
    \begin{proof}
        Let $b_n = \frac{2n}{n + 2}$, and we have $$\lim_{n \to \infty}b_n = \lim_{n \to \infty}\frac{2}{1 + \frac{2}{n}} = \frac{2}{1 + 0} = 2$$

        Let $L = 2$, and by Henie Theorem, $$\lim_{n \to \infty}{a_n} = \lim_{x \to L}{e^x} = e^2$$
        
    \end{proof}

    \subsection*{33. $a_n = \frac{(-1)^n}{2\sqrt{n}}$}

    \begin{proof}
        The sequence converges, and the limit is $0$.


        When $n$ is odd, $$\lim_{n\to \infty}a_n = \lim_{n \to \infty}\frac{-1}{2\sqrt{n}} = 0$$

        And when $n$ is even, $$\lim_{n\to \infty}a_n = \lim_{n \to \infty}\frac{1}{2\sqrt{n}} = 0$$

        So for all $n$, the limit of the sequence is $0$.
    \end{proof}

    \subsection*{38. $\{ \frac{\ln n}{\ln 2n}\}$}
    \begin{proof}
        
        The sequence converges, and the limit is $1$.


        $$\frac{\ln n}{\ln 2n} = \frac{\ln 2n - \ln 2}{\ln 2n} = 1 - \frac{\ln 2}{\ln 2n}$$

        So $$\lim_{n \to \infty}\frac{\ln n}{\ln 2n} = \lim_{n \to \infty}(1 - \frac{\ln 2}{\ln 2n}) = 1 - 0 = 1$$
    \end{proof}

    \subsection*{40. $a_n = \frac{\tan^{-1}n}{n}$}
    The sequence diverges.
    \begin{proof}
        Let $f(x) = \frac{\tan^{-1}x}{x}$, and let $\tan y = x(x \not = 0)$, so $$\frac{\tan^{-1}x}{x} = \frac{y}{\tan y}(y \not = 0)$$

        Let $g(y) = \frac{y}{\tan y}$, and $$\lim_{y \to \infty}g(y) = \frac{\lim_{y \to \infty}y}{\frac{\pi}{2}} = \infty$$

        Because of the relationship of $f(x)$ and $g(y)$, we can get $$\lim_{x \to \infty}f(x) = \infty$$

        So obviously, $$\lim_{n \to \infty}a_n = \lim_{x \to \infty}f(x) = \infty$$
    \end{proof}

    \subsection*{45. $a_n = n \sin(1/n)$}

    \begin{proof}
        
        We first prove $\lim_{x \to 0}\frac{\sin x}{x} = 1$
        

        When $n \to \infty$, $\frac{1}{n} \to 0$

        Let $x = \frac{1}{n}$, so $$\lim_{n \to \infty}n\sin(1/n) = \lim_{n \to \infty}\frac{\sin(1/n)}{1/n} = \lim_{x\to 0}\frac{\sin x}{x} = 1$$

    \end{proof}

    \subsection*{47. $a_n = (1 + \frac{2}{n}) ^n$}

    The sequence converges, and its limit is $e^2$.
    \begin{proof}
        For the definition of the natural constant $e$, we know $$\lim_{n \to \infty}(1 + \frac{1}{n}) ^ n  = e$$

        Let $x = \frac{n}{2}$, and we can get
        $$
        \begin{aligned}
            \lim_{n \to \infty}a_n &= \lim_{n \to \infty}(1 + \frac{1}{\frac{n}{2}})^{\frac{n}{2}+ \frac{n}{2}} \\
            &= \lim_{n \to \infty}((1 + x)^x)^2 = \lim_{n \to \infty}^2(1 + x)^x \\
            &= e^2
        \end{aligned}
        $$
    \end{proof}

    \subsection*{81. Show that the sequence defined by $$a_1 = 1 \qquad   a_{n + 1} = 3 - \frac{1}{a_n}$$ is increasing and $a_n < 3$ for all n. Deduce that $\{ a_n\}$ is convergent and find its limit.}

    \begin{proof}

        $$a_2 = 3 - \frac{1}{a_1} = 3 - 1 = 2$$

        So $a_1 < a_2$

        Suppose that $a_n < a_{n+1}$, then $$a_{n+2} = 3 - \frac{1}{a_{n+1}} > 3 - \frac{1}{a_n} = a_{n + 1}$$

        By the mathematical induction, $a_n$ is increasing.
        
        Because $a_1 = 1 < 3$ and $a_n$ is increasing, $a_n >= a_1 = 1 > 0$

        Also because$$a_{n+1} = 3 - \frac{1}{a_n} < 3$$

        So every term of $a_n$ is less than $3$, which also means $a_n$ is bounded above.

        By the Monotonic Sequence Theorem, $\{ a_n \}$ is convergent.

        Denote the limit of $\{a_n\}$ as $L$.

        Since $a_n \to L$, $a_{n+1} \to L$, by the recurrence formula, we have $$L = 3 - \frac{1}{L}$$

        We can get $$L = \frac{3 + \sqrt{5}}{2} \quad or \quad \frac{3 - \sqrt{5}}{2}$$

        But because $a_n \geq a_1 = 1$, $\frac{3 - \sqrt{5}}{2} < 1$, this solution should be abandoned.
        
        So $$L = \frac{3 + \sqrt{5}}{2}$$

    \end{proof}
        
    \subsection*{82. Show that the sequence defined by $$a_1 = 2 \qquad a_{n+1} = \frac{1}{3 - a_n}$$ satisfies $0 < a_n \leq 2$ and is decreasing. Deduce that the sequence is convergent and find its limit.}
    \begin{proof}
        

        $$a_2 = \frac{1}{3 - a_1} = \frac{1}{3 - 2} = 1 < a_1$$

        so we can get $a_1 < a_2$

        Suppose that $a_n < a_{n + 1}$, then $$a_{n + 2} = \frac{1}{3 - a_{n+1}} < \frac{1}{3 - a_n} = a_{n+1}$$

        By the mathematical induction, $\{a_n\}$ is decreasing.

        Because $\{a_n\}$ is decreasing, $a_n \leq a_1 = 2$, which means $$3 - a_n > 1 > 0$$

        So $$a_{n+1} = \frac{1}{3 - a_n} > 0$$

        So every term of $\{a_n\}$ satisfies $0 < a_n \leq 2$.
        

        By the Monotonic Sequence Theorem, the sequence is convergent.

        Let the limit of $a_n$ be $L$.
        
        Since $a_n \to L$, $a_{n+1} \to L$, and by the recurrence formula, we have $$L = \frac{1}{3 - L}$$

        Solving this equation, we know $$L = \frac{3 + \sqrt{5}}{2} \quad or \quad \frac{3 - \sqrt{5}}{2} $$

        But because $\frac{3 + \sqrt{5}}{2} > \frac{4}{2} = 2$, $a_n \leq 2$, this solution should be abandoned.

        So $$L = \frac{3 - \sqrt{5}}{2}$$

    \end{proof}

    \subsection*{89. Prove that if $\lim_{n \to \infty } a_n = 0$ and $\{b_n\}$ is bounded, then $\lim_{n \to \infty}(a_nb_n) = 0$}

    \begin{proof}
        $\because \{b_n\}$ is bounded, $\therefore \{b_n\}$ has both least upper bound $A$ and greatest lower bound $B$, such that $$A \leq b_n \leq B$$

        So the limit of $b_n$ can only be either $A$ or $B$.

        but no matter what value $b_n$ is, 
        $$
        \begin{aligned}
            \lim_{n\to \infty}(a_nb_n) &= \lim_{n\to \infty}a_n \times \lim_{n\to \infty}b_n \\
            &= 0 \times a \quad or \quad 0 \times b
        \end{aligned}
        $$
        No matter what the bound of $b_n$ is, $$\lim_{n \to \infty}(a_nb_n) = 0$$
    \end{proof}

    \subsection*{91. Let $a$ and $b$ be postive numbers with $a > b$. Let $a_1$ be their arithmetic mean and $b_1$ their geometric mean: $$a_1 = \frac{a + b}{2} \qquad b_1 = \sqrt{ab}$$ Repeat this process so that, in general, $$a_{n+1} = \frac{a_n + b_n}{2} \qquad b_{n+1} = \sqrt{a_nb_n}$$}

    (a) Use mathematical induction to show that $$a_n > a_{n+1} > b_{n+1} > b_n$$

    \begin{proof}

        \subsubsection*{Method 1: mathematical induction}

        When $n = 1$,

        by the mean inequality, we know $$\frac{a+b}{2} \geq \sqrt{ab}$$ only when $a=b$ the euqality holds.

        It also means $$a_1 > b_1, a_2 > b_2$$

        $\because a_2 = \frac{a_1 + b_1}{2} < \frac{a_1 + a_1}{2}, b_2 = \sqrt{a_1 b_1} > \sqrt{b_1b_1}$

        $\therefore$ $$a_1 > a_2 > b_2 > b_1$$

        When $n \geq 2$, suppose when $n = k(k \in N_+)$, we have $$a_k > a_{k + 1} > b_{k + 1} > b_k$$

        Obviously, $$a_{k + 1} > b_{k + 1}, a_{k + 2} > b_{k + 2}$$

        $\because a_{k + 2} = \frac{a_{k + 1} + b_{k + 1}}{2} < \frac{a_{k + 1} + a_{k + 1}}{2}, b_{k + 2} = \sqrt{a_{k + 1}b_{k + 1}} > \sqrt{b_{k + 1}b_{k + 1}} = b_{k + 1}$

        $\therefore$ $$a_{k + 1} > a_{k + 2} > b_{k + 2} > b_{k + 1}$$

        So by the mathematical induction, $$a_n > a_{n + 1} > b_{n + 1} > b_n$$

        \subsubsection*{Method 2: properties of inequality}

        By the mean inequality, we know $$\frac{a+b}{2} \geq \sqrt{ab}$$only when $a=b$ the euqality holds.

        $\because a \not = b \quad  \therefore a_1 > b_1$

        Let $a = a_n, b = b_n$, and we can get $$a_n > b_n, a_{n+1} > b_{n+1}$$

        So $$a_n = \frac{a_n + a_n}{2} > \frac{a_n + b_n}{2} = a_{n+1}$$

        $$b_{n+1} = \sqrt{a_nb_n} > \sqrt{b_nb_n} = b_n$$

        So we can get $$a_n > a_{n+1} > b_{n+1} > b_n$$ for any $n \in N_+$, the inequality holds.
    \end{proof}

    (b) Deduce that both $\{a_n\}$ and $\{b_n\}$ are convergent.

    \begin{proof}
        $\because a_n > a_{n+1} > b_{n+1} > b_n > b_1$

        $\therefore a_n > b_1$, which means $\{a_n\}$ is bounded below.

        $\because \{a_n\}$ is decreasing, $\therefore \{a_n\}$ is convergent.

        $\because a_1 > a_n > a_{n+1} > b_{n+1} > b_n$

        $\therefore b_n < a_1$, which means $\{b_n\}$ is bounded above.

        $\because \{b_n\}$ is increasing, $\therefore \{b_n\}$ is convergent.

    \end{proof}

    (c) Show that $\lim_{n \to \infty} a_n = \lim_{n \to \infty} b_n$, Gauss called the common value of these limits the \textbf{arithmetic-geometric mean} of the numbers $a$ and $b$.

    First let us prove $a_{n + 1} - b_{n + 1} < \frac 1 2(a_n - b_n)$

    \begin{proof}
        $$\begin{aligned}
            a_{n + 1} - b_{n + 1} - \frac 1 2 a_n + \frac 1 2 b_n &= \frac{a_n + b_n}{2} - \frac{a_n}{2} + \frac{b_n}{2} - b_{n+1} \\
            &= b_n - b_{n + 1} < 0
        \end{aligned}$$
        $\therefore a_{n + 1} - b_{n + 1} < \frac 1 2(a_n - b_n)$
    \end{proof}

    With the first conclusion proved above, let us prove the problem (c).
    \begin{proof}
        let $c_n = a_n - b_n$, so we have $$c_{n + 1} < \frac{1}{2}c_n$$

        And then $$c_n < \frac 1 2 c_{n - 1} < \dots < \frac{1}{2^{n - 1}}c_1$$

        $ \because c_n = a_n - b_n \ge 0$

        $\because c_1 = \frac{a + b}{2} - \sqrt{ab} \ge 0$ because when $a>b$ the equality never holds.
        
        Obviously, $\lim_{n \to \infty}\frac{1}{2^{n - 1}}c_1 = 0$

        So $$\lim_{n \to \infty} 0 < \lim_{n \to \infty}c_n < \lim_{n \to \infty}\frac{1}{2^{n - 1}}c_1$$

        By the squeeze theorem, we can get $$\lim_{n \to \infty}c_n = 0$$, which is equivlant to $$\lim_{n \to \infty}a_n = \lim_{n \to \infty}b_n$$
        
    \end{proof}

    \subsection*{92.}
    
    (a) Show that if $\lim_{n \to \infty} a_{2n} = L$ and $\lim_{n \to \infty}a_{2n+1} = L$, then $\{a_n\}$ is convergent and $\lim_{n \to \infty} a_n = L$.

    \begin{proof}
        (1) When $n$ is odd, let $n = 2k + 1$, so $$\lim_{n \to \infty}a_n = \lim_{k \to \infty}a_{2k + 1} = L$$

        (2) When $n$ is evem let $n = 2k$, so $$\lim_{n \to \infty}a_n = \lim_{k \to \infty}a_{2k} = L$$

        Above all, $$\lim_{n \to \infty}a_n = L$$

    \end{proof}
    (b) If $a_1 = 1$ and $$a_{n+1} = 1 + \frac{1}{1 + a_n}$$ find the first eight terms of the sequence $\{a_n\}$. Then use part(a) to show that $\lim_{n \to \infty}a_n = \sqrt{2}$. This gives the \textbf{continued fraction expansion}$$\sqrt{2} = 1 + \frac{1}{2+\frac{1}{2 + \dots}}$$

    $$a_2 = 1 + \frac{1}{1 + a_1} = \frac{3}{2}$$

    $$a_3 = 1 + \frac{1}{1 + a_2} = \frac{7}{5}$$

    $$a_4 = 1 + \frac{1}{1 + a_3} = \frac{17}{12}$$

    $$a_5 = 1 + \frac{1}{1 + a_4} = \frac{41}{29}$$

    $$a_6 = 1 + \frac{1}{1 + a_5} = \frac{99}{70}$$

    $$a_7 = 1 + \frac{1}{1 + a_6} = \frac{239}{169}$$

    $$a_8 = 1 + \frac{1}{1 + a_7} = \frac{577}{408}$$

    \begin{proof}

        $$a_{n+2} = \frac{1}{1 + a_{n+1}} = \frac{1}{1 + 1 + \frac{1}{1 + a_n}} = \frac{a_n + 1}{2a_n + 3} + 1$$



        (1) When $n$ is odd, let $n = 2k + 1$.

        Let $\lim_{k \to \infty}a_{2k + 1} = L$, by $\lim_{k \to \infty}a_{2k + 1} = \lim_{k \to \infty}a_{2k + 3} = L$, so $$L = \frac{1}{1 + L}$$

        Solving the equation, we can get $L = \sqrt{2}$.

        We can know $a_n$ is increasing. and $a_1 = 1 < \sqrt 2$

        So $$\lim_{k \to \infty}a_{2k + 1} = \sqrt{2}$$

        (2) When $n$ is even, let $n = 2k$.

        Let $\lim_{k \to \infty}a_{2k} = L$, by $\lim_{k \to \infty}a_{2k} = \lim_{k \to \infty}a_{2k + 2} = L$, so $$L = \frac{1}{1 + L}$$

        Solving the equation, we can get $L = \sqrt{2}$.

        We can know $a_n$ is decreasing. and $a_2 = \frac 3 2 > \sqrt 2$

        So $$\lim_{k \to \infty}a_{2k} = \sqrt{2}$$

        Above all, $\lim_{n \to \infty}a_n = \sqrt{2}$

    \end{proof}
\end{document}