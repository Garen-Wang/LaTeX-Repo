\documentclass{article}
\usepackage{amsmath}
\usepackage{amsthm}
\usepackage{amssymb}
\usepackage{enumerate}
\usepackage{graphicx}

\begin{document}
    \title{Exercise 11.3}
    \author{Wang Yue from CS Elite Class}
    \date{\today}

    \maketitle

    \subsection*{16. $\sum_{n=1}^\infty \frac{n^2}{n^3+1}$}

    Let $f(x) = \frac{x^2}{x^3+1}$, so $$f'(x) = \frac{2x(x^3+1) - x^2(3x^2)}{(x^3+1)^2} = \frac{x(2-x^3)}{(x^3+1)^2}$$

    $\therefore$ when $x\geq 2, f'(x)<0, f(x)$ is decreasing.

    $$\begin{aligned}
    \int_2^\infty \frac{x^2}{x^3+1}\textrm dx &= \int_2^\infty \frac{1}{3(x^3+1)}\textrm dx^3 \\
    &= \lim_{t\to\infty}\frac{\ln|x^3+1|}{3}\biggl|_2^t \\
    &= \lim_{t\to\infty} \frac{1}{3} \ln |\frac{t^3+1}{9}| \\
    &= \infty
    \end{aligned}$$

    $\because \int_2^\infty \frac{x^2}{x^3+1}\textrm dx$ is divergent

    $\therefore $by integral test, $\sum_{n=2}^\infty \frac{n^2}{n^3+1}$ is also divergent.

    $\therefore \sum_{n=1}^\infty \frac{n^2}{n^3+1}$ is also divergent.

    \subsection*{22. $\lim_{n=2}^\infty \frac{1}{n(\ln n)^2}$}

    Let $f(x) = \frac{1}{x(\ln x)^2}, x \geq 2$, so 
    $$\begin{aligned}
        \int_2^\infty \frac{1}{x(\ln x)^2} \textrm dx &= \int_2^\infty \frac{1}{(\ln x)^2}\textrm d\ln x \\
        &= \lim_{t\to\infty} (-\frac{1}{\ln x}) \biggl|_2^t \\
        &= \lim_{t\to\infty} (\frac{1}{\ln 2} - \frac{1}{\ln t}) \\
        &= \frac{1}{\ln 2}
    \end{aligned}$$

    $\because f(x)$ is obviously decreasing

    $\therefore \int_2^\infty \frac{1}{x(\ln x)^2} \textrm dx $ is convergent

    $\therefore \sum_{n=2}^\infty \frac{1}{n(\ln n)^2}$ is also convergent.

    \subsection*{23. $\sum_{n=1}^\infty \frac{e^{\frac 1 n}}{n^2}$}

    Let $f(x) = \frac{e^{\frac 1 x}}{x^2}, x > 0$, and let $t = \frac 1 x > 0$, then $f(x)$ can be rewritten as $$f(t) = t^2e^t$$

    $\because f'(t) = (t^2+2t)e^t = ((t+1)^2-1)e^t > 0$

    $\therefore \forall t \in (0, \infty), f(t)$ is increasing

    $\therefore \forall x \in (0, \infty), f(x)$ is decreasing

    $$\begin{aligned}
        \because \int_1^\infty \frac{e^{\frac 1 x}}{x^2} \textrm dx &= \int_1^\infty -e^{\frac 1 x}\textrm d\frac 1 x \\
        &= \lim_{t\to\infty}(-e^{\frac 1 x})\biggl|_1^t \\
        &= lim_{t\to\infty} (e^1 - e^{\frac 1 t}) \\
        &= e - 1
    \end{aligned}$$

    which means $\int_1^\infty \frac{e^{\frac 1 x}}{x^2}$ is convergent

    $\therefore \sum_{n=1}^\infty \frac{e^{\frac 1 n}}{n^2}$ is also convergent

    \subsection*{24. $\sum_{n=3}^\infty \frac{n^2}{e^n}$}

    TODO

    \subsection*{26. $\sum_{n=1}^\infty \frac{n}{n^4+1}$}

    Let $f(x) = \frac{x}{x^4+1}, x \geq 1$, so $$f'(x) = \frac{x^4+1 - 4x^4}{(x^4+1)^2} = \frac{1 - 3x^4}{(x^4+1)^2} < 0$$

    $\therefore f(x)$ is decreasing

    $$\begin{aligned}
        \int_1^\infty \frac{x}{x^4+1} \textrm dx &= \int_1^\infty \frac{1}{2(x^4+1)} \textrm dx^2 \\
        &= \lim_{t \to \infty} \frac{\arctan x^2}{2} \biggl|_1^t \\
        &= \lim_{t \to \infty} \frac{\arctan t^2 - \frac \pi 4}{2} \\
        &= \frac \pi 4
    \end{aligned}$$

    $\because \int_1^\infty \frac{x}{x^4+1} \textrm dx$ is convergent

    $\therefore \sum_{n=1}^\infty \frac{n}{n^4+1}$ is also convergent

    \subsection*{33. The Riemann zeta-function $\zeta$ is defined by $$\zeta (x) = \sum_{n=1}^\infty \frac{1}{n^x}$$and is used in number theory to study the distribution of prime numbers. What is the domain of $\zeta$ ?}

    $\because \zeta (x)$ is used to study the distribution of prime numbers

    $\therefore \zeta (x)$ must be integrable

    $\therefore \zeta (x)$ must be convergent

    $\therefore$ by the $p$-series, $x>1$

    $\therefore$ the domain of $\zeta$ is $(1, \infty)$

    \subsection*{Leonhard Euler was able to calculate the exact sum of the $p$-series with $p = 2$:$$\zeta(2) = \sum_{n=1}^\infty \frac{1}{n^2} = \frac{\pi ^2}{6}$$ Use this fact to find the sum of each series.}

    \begin{enumerate}[(a)]
        \item $\sum_{n=2}^\infty \frac{1}{n^2}$
        $$\sum_{n=2}^\infty \frac{1}{n^2} = \sum_{n=1}^\infty \frac{1}{n^2} - \frac{1}{1^2} = \frac{\pi^2}{6} - 1$$

        \item $\sum_{n=3}^\infty \frac{1}{(n+1)^2}$
        $$\begin{aligned}
            \sum_{n=3}^\infty \frac{1}{(n+1)^2} &= \sum_{n=4}^\infty \frac{1}{n^2} \\
            &= (\sum_{n=1}^\infty \frac{1}{n^2}) - \frac{1}{1} - \frac{1}{4} - \frac{1}{9} \\
            &= \frac{\pi^2}{6} - \frac{41}{36}
        \end{aligned}$$

        \item $\sum_{n=1}^\infty \frac{1}{(2n)^2}$
        $$\begin{aligned}
            \because \sum_{n=1}^\infty \frac{1}{n^2} &\textrm{ is convergent} \\
            \therefore \sum_{n=1}^\infty \frac{1}{(2n)^2} &= \sum_{n=1}^\infty \frac{1}{4n^2} \\
            &= \frac 1 4 \sum_{n=1}^\infty \frac{1}{n^2} \\
            &= \frac{\pi^2}{24}
        \end{aligned}$$
    \end{enumerate}


\end{document}