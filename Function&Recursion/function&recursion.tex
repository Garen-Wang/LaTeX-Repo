\documentclass[UTF-8]{beamer}
\usepackage{ctex}
\usetheme{CambridgeUS}
\setCJKsansfont{SimSun}
\setsansfont{Times New Roman}

\begin{document}
\title{函数与递归}
\subtitle{可能是最后一课}
\author{Garen-Wang}
\date{\today}

\titlepage
\begin{frame}{目录}
\tableofcontents
\end{frame}

\begin{section}{函数}
\begin{frame}{函数介绍}
\pause
函数(function),又名子程序。在C语言中,子程序的作用是由一个主函数和若干个函数构成。由主函数调用其他函数,其他函数也可以互相调用。同一个函数可以被一个或多个函数调用任意多次。\pause

函数的基本语法构成由:\textbf{返回类型+函数名+参数+大括号}\pause

函数的实现意义是用来进行专一化处理问题,而不是一个main函数写满了所有基础操作。\pause

使用函数可以更简单明了地解决问题,有时还能办到一些神奇的事情。
\end{frame}
\begin{frame}{常见函数应用}
\pause
函数的应用太广泛了,所以下面的只是一些题目的操作。\pause

\begin{itemize}
  \item 维护进制转换操作
  \item 判断一个字符串是不是回文串
  \item 求出一个函数(数学的)的值
  \item 交换两个数
  \item 求两个数的最大公约数
\end{itemize}
\pause
等等等等。。。
\end{frame}
\end{section}

\begin{section}{函数例题:P1149 火柴棒等式}
\begin{frame}{P1149 火柴棒等式}
\pause
题目要求用$n$根火柴摆出类似$A+B=C$的等式,询问等式种类。注意数字可以是多位数。$n \leq 24$。\pause

加号和等号都需要2根火柴,所以先减掉4,剩下的就是枚举三个数了。\pause

而我们也不用枚举三个数,因为加法的性质,只要确定两个,第三个就出来了。\pause

所以最重要的是判断一个数由几根火柴构成。而实现这个功能,可以使用函数封装起来。\pause

最高不超过1111,使用火柴数最少并且数字最大的数就是它了。\pause

所以一个$O(1111^2)$的双重for循环就能够解决问题。
\end{frame}
\end{section}

\begin{section}{递归}
\begin{frame}{递归}
\pause
递归(recursion)是函数的更深层应用,因为函数允许嵌套使用,甚至可以使用自己!\pause

常见的递归应用方法有:\pause
\begin{itemize}
  \item dfs(深度优先搜索或遍历)
  \item bfs(宽度优先搜过或遍历)
  \item 回溯法
  \item 减小问题范围(递推的反过程)
\end{itemize}
\pause
时间不够,只能够挑一个讲,我们讲回溯法。
\end{frame}
\begin{frame}{回溯法}
\pause
回溯法是一种搜索,相比于传统dfs的不同之处是回溯了。\pause

虽然回溯之后会适当减缓速度,但是也并不慢。\pause

适用于全方位搜索答案的问题。\pause

让我去网上找回溯法的框架。\pause

回溯法的经典例题是\textbf{八皇后问题},时间关系我不讲。
\end{frame}
\end{section}

\begin{section}{递归例题:P1036 选数}
\begin{frame}{P1036 选数}
\pause
题目要求你求出$n$个数取$k$个的排列情况中\textbf{和为素数}的种类数。\pause

显然情况种类数共有$C_n^k$种。但是求出这个东西并没什么卵用。\pause

而我们需要做的是模拟所有选择,然后得到和,再判断是否为质数。\pause

这道题我来亲手写一写。。。
\end{frame}
\end{section}

\begin{section}{后面的话}
\begin{frame}{后面的话}
\pause
How time flies!高一就快结束了!\pause

这个团队在下学期起头应该不会再进行维护了,在复赛后会重新招入新高一的团队成员。\pause

希望大家在这一年能够学到东西,也希望大家多多包涵我的不足之处。\pause

进入高二后,仅存的团队成员可能只会有3到4位。\pause

最希望大家的事情是能够在10月的初赛好好发挥,一个月后公款去广州!(更好的是拿奖了啊!)\pause

我也要去好好准备去招待高一了。。。
\end{frame}
\begin{frame}{吐槽}
\pause
我肝这个课件的时候,CE了一个小时。。。\pause

想写句话来吐槽又CE了。。。\pause

啊!伟大的\LaTeX!
\end{frame}
\end{section}

\begin{frame}
\pause
\begin{center}
  \huge{谢谢!}
\end{center}
\flushright{all made by \LaTeX}
\end{frame}
\end{document}
